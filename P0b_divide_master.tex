
\documentclass[12pt, a4paper]{article}
\usepackage{hyperref}
\hypersetup{
  colorlinks=true,
  linkcolor=blue,
  urlcolor=cyan,
}
\urlstyle{same}
\usepackage[utf8]{inputenc}
\usepackage{amsmath}
\usepackage{amsfonts}
\usepackage{amssymb}
\usepackage{graphicx}


\newtheorem{theorem}{Teorema.}
\newtheorem{lemma}[theorem]{Lema.}
\newtheorem{corollary}[theorem]{Corolario.}
\newtheorem{definition}[theorem]{Definici\'on:}
\newtheorem{example}[theorem]{Ejemplo:}
\newtheorem{problema}[theorem]{Problema:}
\newtheorem{remark}[theorem]{Observaci\'on:}

\usepackage{graphicx}
\usepackage[spanish]{babel}
%\usetheme{default}

\newcommand{\pp}{\mathbb{P}}
\newcommand{\zz}{\mathbb{Z}}
\newcommand{\rr}{\mathbb{R}}
\newcommand{\qq}{\mathbb{Q}}

\usepackage{tikz, tikz-3dplot}

\definecolor{cof}{RGB}{219,144,71}
\definecolor{pur}{RGB}{186,146,162}
\definecolor{greeo}{RGB}{91,173,69}
\definecolor{greet}{RGB}{52,111,72}

\date{}

\begin{document}
\title{Pr\'actico TEOCOMP: Dividir y Conquistar}
\author{Mauricio Velasco}
\maketitle
\begin{enumerate}
\item \emph{(Conteo de inversiones)}
\begin{enumerate}
\item Escriba su implementación en \verb!python! de un algoritmo que retorne el número de inversiones de una lista de enteros calculados por fuerza bruta.
\item Demuestre que la complejidad en tiempo del algoritmo de la parte $(a)$ es $O(n^2)$.
\item Escriba su implementación en \verb!python! del algoritmo recursivo para calcular el número de inversiones.
\item Produzca una tabla comparando los tiempos de ejecucion de sus dos algoritmos para la permutación descendiente $n,n-1,n-2,\dots,1$ para $n=100,1000,10000,100000$.
\end{enumerate}

\item \emph{(Multiplicación de Matrices)}
\begin{enumerate}
\item Escriba el código de una función que implemente el producto de dos matrices $n\times n$ usando la definición aprendida en su curso de álgebra lineal.
\item Demuestre rigurosamente que la complejidad en tiempo de su implementación es $O(n^3)$.
\item Explique por qué es imposible que haya un algoritmo de multiplicación de matrices con time complexity $O(n)$.
\item Demuestre que $P_5+P_4-P_2+P_6 = AE+BG$ en el algoritmo de Strassen visto en clase.
\item Aplique a mano el algoritmo de Strassen utilizándolo para calcular
\[
\left(\begin{array}{cc}
1 & 2 \\
3 & 4 \\
\end{array}\right)
\left(\begin{array}{cc}
5 & 6 \\
7 & 8 \\
\end{array}\right)
=
\]
\end{enumerate}

\item Escriba cuidadosamente el enunciado del Teorema Maestro. Posteriormente aplíquelo para encontrar cotas para las funciones que que cumplen las siguientes desigualdades recursivas.

\begin{enumerate}
\item $T(n)\leq 7T(n/3)+O(n^2)$
\item $A(n)\leq 9A(n/3)+O(n^2)$
\item $B(n)\leq 5B(n/3)+O(n)$
\end{enumerate}

\item Suponga que una función satisface la desigualdad $T(n)\leq T(\lfloor \sqrt{n}\rfloor)+1$ para $n>1$. 
\begin{enumerate}
\item Cuál de las siguientes clases contiene a $T(n)$? ($O(1)$, $O(\log\log(n))$, $O(\log(n))$,$O(\sqrt{n})$).
\item De una demostración de su respuesta al item anterior.
\end{enumerate}

\end{enumerate}
\end{document}

\item Investigue el algoritmo \emph{BubbleSort} para ordenar una lista de $n$ enteros de manera ascendente.
\begin{enumerate}
\item Escriba una descripción en pseudocódigo del algoritmo.
\item Aplíquelo para ordenar la lista $10,9,8,7,6,5,4,3,2,1$ de manera ascendente, mostrando todos los estadios intermedios de la ejecución.
\item Demuestre que el algoritmo es correcto.
\item Demuestre que el número de operaciones es $O(n^2)$.
\end{enumerate}

\item Considere la siguiente modificación de \emph{MergeSort} dividiendo el input en tres tercios y no en dos mitades.
\begin{enumerate}
\item Escriba el pseudocódigo de la función \emph{merge} en este caso y
\item Estime y demuestre una cota superior para el tiempo de ejecución en un input de longitud $n=3^m$.
\end{enumerate}

\item Te dan un arreglo de $n$ números distintos donce $n=2^m$ para algun entero $m$. Propón un algoritmo que identifica el segundo número más grande del arreglo usando a lo más $n+\log(n)-2$ comparaciones. Demuestra su validez y verifica formalmente que tiene el número correcto de comparaciones (Sugerencia: Qué información queda después de calcular el número más grande?).

\item Implemente en \verb! python! el algoritmo de Karatsuba para multiplicar enteros y utilícelo para calcular el cuadrado del número entero más pequeño que no se puede representar mediante una variable de tipo \verb!int! en python. Escriba el código de su implementación, explique como encontró el entero y escriba el producto resultante.

\item Ordene las siguientes funciones en orden de tasa de crecimiento creciente demostrando su respuesta (es decir $g(n)$ esta despues de $f(n)$ si $f=O(g(n))$) 
\begin{enumerate}
\item $\sqrt{n}$
\item $10^n$
\item $n^{1.5}$
\item $n^{\frac{5}{3}}$
\item Use \verb! python! para graficar las funciones y estimar los valores $n_0$ para los que las desigualdades que encontró se cumplen. 
\end{enumerate}


\item Ordene las siguientes funciones en orden de tasa de crecimiento creciente demostrando su respuesta (es decir $g(n)$ esta despues de $f(n)$ si $f=O(g(n))$) 
\begin{enumerate}
\item $n^2\log_2(n)$
\item $2^n$
\item $2^{2^n}$
\item $n^{\log_2(n)}$
\item $n^2$
\end{enumerate}


\end{enumerate}

\end{document}