
\documentclass[12pt, a4paper]{article}
\usepackage{hyperref}
\hypersetup{
  colorlinks=true,
  linkcolor=blue,
  urlcolor=cyan,
}
\urlstyle{same}
\usepackage[utf8]{inputenc}
\usepackage{amsmath}
\usepackage{amsfonts}
\usepackage{amssymb}
\usepackage{graphicx}


\newtheorem{theorem}{Teorema.}
\newtheorem{lemma}[theorem]{Lema.}
\newtheorem{corollary}[theorem]{Corolario.}
\newtheorem{definition}[theorem]{Definici\'on:}
\newtheorem{example}[theorem]{Ejemplo:}
\newtheorem{problema}[theorem]{Problema:}
\newtheorem{remark}[theorem]{Observaci\'on:}

\usepackage{graphicx}
\usepackage[spanish]{babel}
%\usetheme{default}

\newcommand{\pp}{\mathbb{P}}
\newcommand{\zz}{\mathbb{Z}}
\newcommand{\rr}{\mathbb{R}}
\newcommand{\qq}{\mathbb{Q}}

\usepackage{tikz, tikz-3dplot}

\definecolor{cof}{RGB}{219,144,71}
\definecolor{pur}{RGB}{186,146,162}
\definecolor{greeo}{RGB}{91,173,69}
\definecolor{greet}{RGB}{52,111,72}
\date{}


\begin{document}
\title{PRÁCTICO : Máquinas de Turing}
\author{Mauricio Velasco}
\maketitle{}
\begin{enumerate}
\item \begin{enumerate}
\item Diseñe y escriba una máquina de Turing que escanea hacia la derecha hasta que encuentra dos $a$'s consecutivas y luego se detiene. El alfabeto de la máquina debe ser $\Sigma =\{a,b,\cup,\triangle\}$ y debe dar la descripción de la máquina en completo detalle (como $5$-tupla).
\item Escriba las configuraciones que ocurren al ejecutar su máquina con input $\cup bbabaa$.
\end{enumerate}


\item Dé un ejemplo de una máquina de Turing sobre el alfabeto $\{a\}$ con un solo halting state $h$ que cumpla:
\begin{enumerate}
\item Si la máquina se inicia con la palabra $_aaaaaa\dots a$ donde la $a$ aparece un número par de veces y la cabeza lectora en el vacío inicial entonces la máquina se detiene en estado $h$ con la cabeza en el vacío inicial.
\item Si la máquina se inicia con la palabra $_aaaaaa\dots a$ donde la $a$ aparece un número \emph{impar} de veces y la cabeza lectora en el vacío inicial entonces la máquina NO se detiene (es decir continua realizando operaciones y movimientos y nunca llega al estado $h$).
\end{enumerate}
Demuestre que su máquina cumple las características pedidas.


\item Construya una máquina de Turing (usando, si quiere, la notación de máquinas de Turing jerárquicas) que calcule la funcion $f: \{a,b\}^*\rightarrow \{a,b\}^*$ dada por $f(w)=ww^R$ donde $w^R$ significa la palabra reversa a $w$. Muestre la ejecución de la misma en una cadena representativa. (nota: Puede asumir que la cinta inicia con la palabra $_w$ y que la cabeza lectora se encuentra en el vacío inicial).


\item Describa una máquina de Turing que semidecida el lenguaje $a^*ba^*b$.

\item Utilice máquinas de Turing no deterministas para demostrar que:
\begin{enumerate}
\item La clase de lenguajes recursivos esta cerrada bajo unión, concatenación y estrella de Kleene.
\item La clase de lenguajes recursivamente enumerables esta cerrada bajo unión, concatenación y estrella de Kleene.

\end{enumerate}



\end{enumerate}




\end{document}